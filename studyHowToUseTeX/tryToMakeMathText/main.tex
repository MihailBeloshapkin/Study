\documentclass{article}
\usepackage[utf8]{inputenc}
\usepackage{mathtext}
\usepackage{natbib}
\usepackage{amsmath}
\usepackage{mathtools}
\usepackage{mathptmx}
\usepackage{mathrsfs}
\usepackage{amsfonts}
\usepackage{amssymb}
\usepackage{graphicx}
\usepackage[english,bulgarian,ukranian,russian]{babel}
\linespread{1.5}

\usepackage[left=1.5cm, right=1.5cm, top=2cm, bottom=2cm, bindingoffset=0cm]{geometry}
\setlength{\parskip}{0.1cm}
\setlength{\parindent}{0.0cm}
\begin{document}

\large
\begin{center}
    \underline{Этот документ создан для тренировки верстки текстов, содержащих математические символы}
\end{center}
Катеты и гиппотенуза в прямоугольном треугольнике связаны следующим соотношением:
\[ a^2 + b^2 = c^2\] где $a$ и $b$ - это катеты, а $c$ - гипотенуза\\
Пусть $V$ - это линейное пространство\\
$W \subset V$ называется линейным подпространством в $V$, если:\\

1) $W \neq \varnothing$\\
2) $\forall w, w' \in W: w + w' \in W$\\
3) $\forall w \in W, \forall \alpha \in K: \alpha w \in W$\\
\underline{Св-ва:} Пусть $W$ - линейное подпространство в V\\
1) $0 \in W$\\
2) $w \in W \Rightarrow -w \in W$\\
\underline{Док-во:}\\
2) $-w = -(1w) = (-1)w \in W$\\
1) $w \in W \Rightarrow -w \in W; w - w = 0$\\

\underline{Опр} Пусть $V$ -- линейное пространство над полем $K$, $X \subset V$\\
Линейной оболочкой $X$ называется:\\
\[Lin(X) = \left\{ \sum\limits_{i=1}^n \alpha_i\cdot V_i | n \geqslant 1, \alpha_i \in K \right\} \]\\
\underline{Опр:} Набор векторов $V_1, \ldots v_m \in v$ называется порождающим семейством, если $Lin(v_1, \ldots v_m) = V$\\
Пусть $v_1, \ldots v_{n+1} \in v$ -- система образующих $v$\\
Тогда $v_1, \ldots v_m \text{- система образующих} V \Leftrightarrow v_{n + 1}\in Lin(v_1, \ldots v_m)$ \\
\underline{Док-во:}  \\
\begin{math}
\Rightarrow  v_{n + 1} \in V = Lin(v_1, \ldots v_m) \\
\Leftarrow \text{Пусть} v_{m + 1} \in Lin(v_1, \ldots v_m) \\ \text{Пусть} v \in V \\ Lin(v_1, \ldots v_{m + 1}) = V 
\Rightarrow v = \alpha_1v_1 + \ldots + \alpha_{m + 1}v_{m + 1} \text{ для некоторого } \alpha_1\ldots\alpha_{m + 1} \in K\\
\text{ Но } v_{m + 1} = \beta_1 v_1 + \ldots + \beta_m v_m \text{ для некоторого } \beta_1 \ldots \beta_m \in K 
\Rightarrow v = \alpha_1 v_1 + \ldots + \alpha_m v_m  + \alpha_{m + 1}(\beta_1 v_1 + \ldots + \beta_m v_m)
\end{math} \\
Некоторые интегралы: \\
$$\int f(x)dx = F(x) + C, \text{ где } F'(x) = f(x), \text{ а } C \in \mathbb{R}$$ 
$$\int x^n dx = \frac{x^{n + 1}}{n + 1}$$ 
$$\int e^x = e^x$$
$$\int \sin(x)dx = -\cos(x)$$
$$\int \frac{dx}{cos^2(x)} = \tg(x)$$
$$\int \frac{dx}{\sqrt{a^2 + x^2}} = \frac{1}{a} \arctan(\frac{x}{a})$$



\[\left(\begin{array}{cccccccc}
1 & 2 & 3 & 4 & 5 & 6 & 7 & 8\\
8 & 7 & 6 & 5 & 4 & 3 & 2 & 1
\end{array}\right)
\]
\begin{equation*}
\delta_{ij} =
\left\{
\begin{array}{lr}
1 & \text{ для } i = j\\
0 & \text{ для } i \ne j
\end{array}
\right.
\end{equation*}\\
\large



\end{document}
