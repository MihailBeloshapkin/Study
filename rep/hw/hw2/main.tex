\documentclass{article}
\usepackage[utf8]{inputenc}
\usepackage{mathtext}
\usepackage{natbib}
\usepackage{amsmath}
\usepackage{mathtools}
\usepackage{mathptmx}
\usepackage{mathrsfs}
\usepackage{amsfonts}
\usepackage{amssymb}
\usepackage{graphicx}
\usepackage{xcolor}
\usepackage{hyperref}
\usepackage[english, russian]{babel}
\linespread{1.6}
\definecolor{linkcolor}{HTML}{750B80}
\definecolor{urlcolor}{HTML}{750B80}
 
\hypersetup{pdfstartview=FitH,  linkcolor=linkcolor,urlcolor=urlcolor, colorlinks=true}


\usepackage[left=1.5cm, right=1.0cm, top=2cm, bottom=2cm, bindingoffset=0cm]{geometry}
\setlength{\parskip}{0.1cm}
\setlength{\parindent}{0.0cm}

\begin{document}
\large
\begin{center}
    дз
\end{center}
\begin{flushleft}
\underline{На 9 ноября 2020г.}
\end{flushleft}
1. Найди область определения:
\[ f(x) = \log_{\ln x} (x^2 + x - 20)\]
\[ f(x) = \log_{\sin x} \cos x\]
2.Решить уравнение 
\[ \log_3 (x^2 - 2x) = 1 \]
И найти корни, принадлежащие отрезку \( \left[ \log_2 0,2; \log_2 5 \right] \) \\
3. Решить уравнение 
\[ 6 \log^2_8 x - 5\log_8 x + 1 = 0 \]
И найти корни, принадлежащие отрезку \( \left[ 2; 2.5 \right] \) \\
4. Решить уравнение
\[ \cos 2x + \sin^2 x = 0,5 \]
И указать корни, принадлежащие отрезку \( [-\frac{7\pi}{2}; -2\pi] \) \\
5. Решить уравнение
\[ \frac{3^{\cos x}}{9^{\cos^2 x}} = 4^{2 \cos^2 x - \cos x} \]
И указать корни, принадлежащие отрезку \( [-\frac{3\pi}{2}; \frac{\pi}{6}] \) \\
Напоминаю, что решения лучше присылать единым pdf файлом, объединять много файлов в один можно тут: \\
\href{https://pdf.io/ru/}{https://pdf.io/ru/}

\end{document}