\documentclass{article}
\usepackage[utf8]{inputenc}
\usepackage{mathtext}
\usepackage{natbib}
\usepackage{amsmath}
\usepackage{mathtools}
\usepackage{mathptmx}
\usepackage{mathrsfs}
\usepackage{amsfonts}
\usepackage{amssymb}
\usepackage{graphicx}
\usepackage[english, russian]{babel}
\linespread{1.4}

\usepackage[left=1.5cm, right=1.5cm, top=2cm, bottom=2cm, bindingoffset=0cm]{geometry}
\setlength{\parskip}{0.1cm}
\setlength{\parindent}{0.0cm}

\begin{document}
\large
\begin{flushleft}
\underline{9 ноября 2020 г.}
\end{flushleft}
С уравнениями познакомились -- теперь пришло время неравенств. Задачи данного типа не отличаются
какой-то особой сложностью, поэтому при достаточно уверенной теоретической базе научиться решать
их не составит особого труда. Но нужно помнить об одной очень важной для ЕГЭ вещи --  а именно об аккуратности
и педантичности, так как подавляющее большинство ошибок совершаются именно по невнимательности
будущего абитуриента. 
\section*{Неравенства}
Прежде чем приступить к решениям и разборам, нужно уточнить один очень важный факт:
\[ \log_a x \text{ строго убывает  при } a \in (0;1)\]
\[ \log_a x \text{ строго возрастает при } a \in (1;\infty)\]
Ну а вот первая задача: \\
1. Решить неравенство:
\[ \log_{x + 2} (x^2 - 3x) \geq \log_{x + 2}(-x)\]
Переменная находится в обоих частях логарифма. Благо случай достаточно легкий -- основание-то
одинаковое. Воспользуемся этим и перенесем все в одну часть:
\[ \log_{x + 2} (x^2 - 3x) - \log_{x + 2}(-x) \geq 0 \Leftrightarrow \]
\begin{equation*}
\Leftrightarrow
    \begin{cases}
    \log_{x + 2} \frac{x^2 - 3x}{-x} \geq 0 (*)\\
    x < 0 \\
    x^2 - 3x > 0
    \end{cases}
    \Leftrightarrow
    \begin{cases}
    \log_{x + 2} (-x + 3) \geq 0 (*)\\
    x < 0 \\
    x^2 - 3x > 0 \\
    x \neq 0
    \end{cases}
    (0)
\end{equation*}
Решим неравенство (*). Представим в виде системы:
\begin{equation*}
    \left[
    \begin{gathered}
      \begin{cases}
          -x + 3 \geq 1 \\
          x + 2 > 1
      \end{cases} \\
      \begin{cases}
          -x + 3 \leq 1 \\
          x + 2 < 1
      \end{cases}
    \end{gathered}
    \right.
    \Leftrightarrow
    \left[
    \begin{gathered}
      \begin{cases}
          2 \geq x \\
          x > -1
      \end{cases} \\
      \begin{cases}
          2 \leq x \\
          x < -1
      \end{cases}
    \end{gathered}
    \right.
    \Leftrightarrow
    x \in (-1;2] 
\end{equation*}
\newpage
Сопоставим с первоначальной системой (0):
\begin{equation*}
\Leftrightarrow
    \begin{cases}
    x \in (-1;2] \\
    x + 2 \in (0; 1) \cup (1;+\infty) \\
    x < 0 \\
    x^2 - 3x > 0 \\
    x \neq 0
    \end{cases}
    \Leftrightarrow
    \begin{cases}
    x \in (-1;2] \\
    x \in (-2; -1) \cup (-1;+\infty) \\
    x < 0 \\
    x \in (-\infty;0) \cup (3;+\infty) \\
    x \neq 0
    \end{cases}
    \Leftrightarrow
    x \in (-1;0)
\end{equation*}
2. Решить неравенство:
\[ \log_{x - 2} 5 \leq 2\]
3. Решить неравенство 
\[ x^{2 - 4\log_2 x + \log^2_2 x} < \frac{1}{x} \]
Разбор задачи в конце документа! \\
4. Решить неравенство:
\[ \frac{\log^2_{\frac{1}{2}} - 4}{\log_{\frac{1}{2}} - 1} \leq 0\]
5. Решить неравенство:
\[ 4 \cdot 3^x \log_{\frac{1}{2}} x + 36 \log_2 x + 9 \geq 3^x \]
6. Решить неравенство
\[ \frac{\log_2 (3\cdot2^{x - 1} - 1)}{x} \geq 0 \]
7. Решить неравенство:
\[ \log_x 2 \cdot \log_{2x} 2 \cdot \log_{2} (16x) \geq 1 \]
\(\mathbf{Fact:}\) Отметим одну существенную вещь, а именно попробуем решить в общем
случае неравенство вида:
\[ \log_{g(x)} f(x) > 0 \]
Мы знаем, что \( \log_{g(x)} f(x) \text{ строго убывает  при } g(x) \in (0;1),  
\text{ строго возрастает при g(x) > 1 }\). Также любой логарифм принимает
нулевой значение только при $f(x) = 1$. Не забывая об ОДЗ, запишем это
в виде системы:
\begin{equation*}
    \left[
    \begin{gathered}
    \begin{cases}
    f(x) > 1 \\
    g(x) > 1
    \end{cases}
    \\
    \begin{cases}
    f(x) > 1 \\
    g(x) < 1
    \end{cases}
    \end{gathered}
    \right.
    \Leftrightarrow
    (f(x) - 1) \cdot (g(x) - 1) > 0
\end{equation*}
То есть для решения изначального неравенства достаточно решить систему:
\begin{equation*}
    \begin{cases}
    (f(x) - 1) \cdot (g(x) - 1) > 0 \\
    f(x) > 0 \\
    g(x) > 0 \\
    g(x) \neq 1
    \end{cases}
\end{equation*}
\section*{Тригонометрия}
Буквально пару слов про обратные тригонометрическиее функции.
\[ \arcsin x + \arccos x = \frac{\pi}{2} \]
\[ \arctan x + \arcctg x = \frac{\pi}{2} \]
Док-во:
\[ \arcsin x = \arcsin (\cos (\arccos x)) = \arcsin (\sin (\frac{\pi}{2} - \arccos x))
= \frac{\pi}{2} - \arccos x\]
\[ \arctan x = \arctan (\ctg (\arcctg x)) = \arctan (\tan(\frac{\pi}{2} - \arcctg x))
 = \frac{\pi}{2} - \arcctg x \]
1. Решить неравенство:
\[ \frac{3\ctg x}{1 + \ctg x} \leq 2\]
2. Решить неравенство:
\[ \arccos x < \arcsin x \]
3. Решить уравнение:
\[ \sqrt{\sin^2 x  - \cos^2 x} \left( \frac{1}{\ctg 2x} + 1\right) = 0\]
4. Решить уравнение
\[ 2\cdot \cos 2x + 4 \cos(1.5\pi - x) + 1 = 0 \]
И найти корни, принадлежащие отрезку $[1.5\pi; 3\pi]$
\\
5. Решить уравнение:
\[ \arcsin(\cos(2\arcctg x)) = 0 \]
\newpage
2. (Решение) 
\[ x^{2 - 4\log_2 x + \log^2_2 x} < \frac{1}{x} \Leftrightarrow \]
\[ \Leftrightarrow x^{2 - 4\log_2 x + \log^2_2 x} < x^{-1} \]
Как мы знаем, показательная функция $a^x$ возрастает при $a > 1$,
убывает при $0 < a < 1$, и константна при $a = 1$. Так как неравенство строгое,
последний вариант отпадает. Рассмотрим первые два из них: \\
\begin{equation*}
\left[
\begin{gathered}
\begin{cases}
x > 1 \\
2 -4\log_2 x + \log^2_2 x < -1
\end{cases}
\\
\begin{cases}
x \in (0; 1) \\
2 -4\log_2 x + \log^2_2 x > -1
\end{cases}
\end{gathered}
\right.
\Leftrightarrow
\left[
\begin{gathered}
\begin{cases}
x > 1 \\
(\log_2 x - 3) \cdot (\log_2 x - 1) < 0
\end{cases}
\\
\begin{cases}
x \in (0; 1) \\
(\log_2 x - 3) \cdot (\log_2 x - 1) > 0
\end{cases}
\end{gathered}
\right.
\end{equation*}
\\
Что делать дальше, я думаю понятно.
\\
4. (Решение) \\
\[ 4 \cdot 3^x \log_{\frac{1}{2}} x + 36 \cdot \log_2 x + 9 \geq 3^x \Leftrightarrow 
 -4 \cdot 3^x \log_2 x + 36 \log_2 x + 9 - 3^x \geq 0 \Leftrightarrow\]
\[ 4 \log_2 x (9 - 3^x) + (9 - 3^x) \geq 0 \Leftrightarrow 
(4\log_2 x + 1) \cdot (9 - 3^x) \geq 0\]
\end{document}