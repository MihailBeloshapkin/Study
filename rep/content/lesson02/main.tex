\documentclass{article}
\usepackage[utf8]{inputenc}
\usepackage{mathtext}
\usepackage{natbib}
\usepackage{amsmath}
\usepackage{mathtools}
\usepackage{mathptmx}
\usepackage{mathrsfs}
\usepackage{amsfonts}
\usepackage{amssymb}
\usepackage{graphicx}
\usepackage[english, russian]{babel}
\linespread{1.6}

\usepackage[left=1.5cm, right=1.5cm, top=2cm, bottom=2cm, bindingoffset=0cm]{geometry}
\setlength{\parskip}{0.1cm}
\setlength{\parindent}{0.0cm}

\begin{document}
\large
\begin{flushleft}
\underline{2 ноября 2020 г.} 
\end{flushleft}
\paragraph{}
Тема сегодняшней пары -- логарифмические, показательные, тригонометрические уравнения и неравенства.
Данные задачи из второй части ЕГЭ не являют собой какую-то неразрешимую проблему, тем более что последние несколько
лет наблюдается тенденцию упрощения задачи 13 и 15, что, собственно, не очень хорошо. Но сегодня не об этом --
предлагаю перейти к практике. 
 
\section{Логарифм}
\paragraph{}
Первое, что стоит отметить, так это то, что логарифм имеет очень строгое ОДЗ. Выражение  \( \log_a b\) определено
только тогда, когда \underline{\( b > 0 \) и \( a \in (0; 1) \cup (1;+\infty) \)} \\
Найти область определения: \\
\[ f(x) = \log_{x^2 - 2x + 2} (\sqrt{1 - x^2})\]
\[ f(x) = \log_{\sin x} \left((x - \frac{7\pi}{30})(\frac{37\pi}{31} - x)\right)\]
\[ f(x) = \log_{\ln x} (x^2 + x - 20)\]
\[ f(x) = \log_{\sin x} \cos x\]
Разберу только первый. Выпишем ОДЗ в виде системы неравентсв: 
\begin{equation*}
    \begin{cases}
    \sqrt{1 - x^2} > 0 \\
    x^2 - 2x + 2 > 0 \\
    x^2 - 2x + 2 \neq 1
    \end{cases}
    \Leftrightarrow
    \begin{cases}
    (1 - x)(1 + x) > 0\\
    (x - 1)^2 + 1 > 0 \text{ -- верно } \forall x \\
    (x - 1)^2 \neq 0
    \end{cases}
\end{equation*}
Вот такая системка. Отсюда легко получить окончательный ответ
\begin{equation*}
    \begin{cases}
    x \in (-1; 1) \\
    x \in \mathbb{R}\\
    x \neq 1
    \end{cases}
    \Leftrightarrow
    x \in (-1; 1)
\end{equation*}
Остальные примеры немного более показательные, поэтому рекомендованы к решению. \\
\newpage
Время перчислить свойства \\
\(\mathbf{Properties:}\)
\[ \log_a (x \cdot y) = \log_a x + \log_a y \]
\[ \log_a \frac{x}{y} = \log_a x - \log_a y \]
\[ \log_a x^n = n \cdot \log_a x \]
\[ \log_{y^n} x = \frac{1}{n} \log_y x \] 
\[ \log_y x = \frac{\log_z x}{\log_z y} \]
\[ \log_y x = \frac{1}{\log_x y} \]
\[ z^{\log_y x} = x^{\log_y z} \]
\[ x^{log_x y} = y \]
\(\mathbf{Fact:}\) Логарифмическая является обратной к показательной.
В самом деле, пусть у нас есть \\ \( f(x) = \log_a x \text{ и } g(x) = a^x \). Тогда 
\[ g(f(x)) = a^{\log_a x}  = x \]
\[ f(g(x)) = \log_a a^x = x \log_a a = x \cdot 1 = x \]
При этом \( E_f = \mathbb{R}, D_f = \mathbb{R}_+\) и 
\( E_g = \mathbb{R}_+, D_g = \mathbb{R}\). Что и требовалось доказать \\ \\
\(\mathbf{Fact:}\) \( f(x) = \log_a b \) Тогда если \( a > 1 \), то 
\( f(x) \) возрастает на $\mathbb{R}_+$, если \( a \in (0; 1) \), то
\( f(x) \) убывает на $\mathbb{R}_+$. \\
Действительно, как мы знаем, обратные функции сохраняют монотонность исходной.
\( a^x \) возрастает при $a > 1$ и убывает при \( a \in (0; 1) \). Из этого получаем
монотонность логарифма. \\ \\
А теперь к задачам: \\
1. Решить уравнение 
\[ \log_5 (2 - x) = \log_{25} x^4 \]
И найти корни, принадлежащие отрезку \( \left[ \log_9 \frac{1}{82}; \log_9 8 \right] \) \\

2. Решить уравнение 
\[ 4^x - 2^{x + 3} + 15 = 0 \]
И найти корни, принадлежащие отрезку \( \left[ 2; \sqrt{10} \right] \) \\
3. Решить уравнение 
\[ 6 \log^2_8 x - 5\log_8 x + 1 = 0 \]
И найти корни, принадлежащие отрезку \( \left[ 2; 2.5 \right] \) \\
4. Решить уравнение 
\[ 1 + \log_2 (9x^2 + 5) = \log_{\sqrt{2}}\sqrt{8x^4 + 14} \]
И найти корни, принадлежащие отрезку \( \left[ -1; \frac{8}{9} \right] \) \\
5. Решить уравнение 
\[ \log_7 (x + 2) = \log_{49} x^4 \]
6. Решить уравнение
\[ 4^{x^2 - 2x + 1} + 4^{x^2 - 2x} = 20 \]
7. Решить уравнение 
\[ \log_3 (x^2 - 2x) = 1 \]
И найти корни, принадлежащие отрезку \( \left[ \log_2 0,2; \log_2 5 \right] \)
\section{Тригонометрия}
\paragraph{}
Ну тут особо комментировать нечего: больше практики -- лучше результат. Поэтому: \\
1. Решить уравние 
\[ \sqrt{3}\tan(5\pi + 2x) = 3\]
И указать корни, принадлежащие отрезку \( [\pi; \frac{5\pi}{2}] \) \\
2. Решить уравнение
\[ 9 \cdot 81^{\cos x} - 28 \cdot 9^{\cos x} + 3 = 0 \]
И указать корни, принадлежащие отрезку \( [\frac{5\pi}{2}; 4\pi] \) \\
3. Решить уравнение
\[ \sqrt{2}\sin^2 x = \cos\left( \frac{3\pi}{2} - x\right) \]
И указать корни, принадлежащие отрезку \( [2\pi; \frac{7\pi}{2}] \) \\
\newpage
4. Решить уравнение 
\[\cos 2x = \sin\left( x + \frac{\pi}{2}\right)\]
И указать корни, принадлежащие отрезку \( (-2\pi; -\pi)\) \\
5. Решить уравнение
\[ 4 \cos^4 x - 4\cos^2 x + 1 = 0 \]
И указать корни, принадлежащие отрезку \( (-2\pi; -\pi) \) \\
6. Решить уравнение
\[ \cos 2x + \sin^2 x = 0,5 \]
И указать корни, принадлежащие отрезку \( [-\frac{7\pi}{2}; -2\pi] \) \\
 7.Решить уравнение
\[ \cos 2x -3\cos x + 2 = 0 \]
И указать корни, принадлежащие отрезку \( [-4\pi; \frac{5\pi}{2}] \) \\
8. Решить уравнение
\[ \frac{3^{\cos x}}{9^{\cos^2 x}} = 4^{2 \cos^2 x - \cos x} \]
И указать корни, принадлежащие отрезку \( [-\frac{3\pi}{2}; \frac{\pi}{6}] \) \\

\end{document}