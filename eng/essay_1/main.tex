\documentclass[14pt]{article}
\usepackage[utf8]{inputenc}


\usepackage{natbib}
\usepackage{graphicx}
\usepackage[left=2cm,right=2cm, top=2cm,bottom=2cm,bindingoffset=0cm]{geometry}
\setlength{\parskip}{0.05cm}

\begin{document}

\large

\section*{The role of museums in society life today}
\paragraph{}
Museums are really important part of the general education: here we can explore 
almost all periods of the history, starting with WW1 or WW2 and ending with ancient Greece.
And do modern museums meet these conditions? Here we have two text with two different opinions.
\paragraph{}
The first author explains that the present requires some changes in the museum policy: museums 
have to be community-oriented nowadays. Curators should choose another way to present exhibitions.
For example, modern technologies give an opportunity to add interactive component to breathe more life into exhibits.
 Perhaps the author wants to say that modern people are not enough educated to attend museums which only 
contain exhibits, and modern technologies will increase number of people visiting them.
\paragraph{}
The second author is not as positive as the first: he complains that the museum is mostly a kind of
entertainment, and it does not give any deep knowledge after visiting: "the medium has become more important
 than the message" -- he explains. The aim of the museums existence is to broaden the horizons of
its visitors, but the information which is given there nowadays is superficial, and as a result high quality studying
is replaced by technical effects.
\paragraph{}
The first author thinks that museums have chosen right way, and the addition of modern technologies could
turn museum into a place where almost everyone will have an opportunity to spend his leisure time.
The second author suggests that "technological wizardry" will reduce the quality of education.
\end{document}
