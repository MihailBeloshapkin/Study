\documentclass{article}
\usepackage[utf8]{inputenc}
\usepackage{mathtext}
\usepackage{natbib}
\usepackage{amsmath}
\usepackage{mathtools}
\usepackage{mathptmx}
\usepackage{mathrsfs}
\usepackage{amsfonts}
\usepackage{amssymb}
\usepackage{graphicx}
\usepackage[english, russian]{babel}
\linespread{1.5}

\usepackage[left=1.5cm, right=1.5cm, top=2cm, bottom=2cm, bindingoffset=0cm]{geometry}
\setlength{\parskip}{0.1cm}
\setlength{\parindent}{0.0cm}

\begin{document}
\large
\begin{center}
    \underline{дз по алгебре}
\end{center}
\underline{3.1} \\
a) Найти многочлен $p$ не выше четвертой степени т.ч $p(-4) = 5, p(-3) = 14, p(-1) = 2, \\ p(1) = 10, p(-2) = -5$ \\
Воспользуемся формулой Лагранжа для интерполяционной задачи: 
\[ f = \sum\limits_{j=1}^n y_j\frac{\sideset{}{_{i \neq j}}\prod(X - x_i) }{\sideset{}{_{i \neq j}}\prod(x_j - x_i)}\]
\underline{2.1} \\
Перестановка:
\[ \pi = \left(\begin{array}{cccccccc}
    1 & g_1 & 3 & g_2 & 5 & g_3 & g_4 & g_5 \\
    1 & \pi(g_1) & 3 & \pi(g_2) & 5 & \pi(g_3) & \pi(g_4) & \pi(g_5)
\end{array}\right)\]
Ассоциативность очевидна в виду того, что это по прежнему композиция перестановок. Руйтральный элемент: \\
\[e = \left( \begin{array}{cccccccc}
    1 & 2 & 3 & 4 & 5 & 6 & 7 & 8 \\
    1 & 3 & 3 & 4 & 5 & 6 & 7 & 8
\end{array}\right)\]
Обратная перестановка к перестановне $\pi$: \\
\[ \pi^{-1} = \left( \begin{array}{cccccccc}
     1 & \pi(g_1) & 3 & \pi(g_2) & 5 & \pi(g_3) & \pi(g_4) & \pi(g_5) \\
     1 & g_1 & 3 & g_2 & 5 & g_3 & g_4 & g_5
\end{array} \right)\] \\
\[ \pi\pi^{-1} = \pi^{-1}\pi = e \]
Значит это группа
\end{document}