\documentclass[14pt,pdf,hyperref={unicode}]{beamer}
\usepackage{amsmath,amsthm,amssymb}
\usepackage{mathtext}
\usepackage[T1,T2A]{fontenc}
\usepackage[utf8]{inputenc}
\usepackage[english,bulgarian,ukranian,russian]{babel}

\usepackage{natbib}
\usepackage{graphicx}



\setbeamercovered{dynamic}
\title{Сравнение платформ .NET Core, .NET Framework, Mono}
\author{Белошапкин Михаил, 144 группа}

\usepackage{natbib}

\usepackage{graphicx}

\begin{document}

\maketitle

\begin{frame}{Платформа .net Framework}
    \begin{itemize}
        \item Самая старая
        \item Наиболее популярна
        \item Поставляется в комплекте с Windows
        \item Не являеся платформой с открытым ИК
    \end{itemize}
\end{frame}

\begin{frame}{Платформа .net Framework}
    \begin{itemize}{}
    \item Компилятор языка C\#
    \item Среда исполнения
    \begin{itemize}{}
        \item Common Language Runtime
        \item Компилятор CLI (Just-In-Time)
        \item Загрузчик сборок
        \item Сборщик мусора
        \item Подсистемы управления многопоточностью
    \end{itemize}
     \item Библиотека классов
        \begin{itemize}
            \item Консольные приложения, Windows Froms, ASP.NET
        \end{itemize}
\end{itemize}
\end{frame}

\begin{frame}{Преимущества .net Framework}
\begin{itemize}
    \item Не все сторонние библиотеки и пакеты NuGet доступны .NET Core
    \item .NET технологии, не доступные .NET Core
    \begin{itemize}
        \item Некоторые компоненты ASP.NET, службы рабочих процессов, поддержка языков
    \end{itemize}
\end{itemize}
    
\end{frame}


\begin{frame}{Платформа .NET Core}
\begin{itemize}
    \item На основе .NET Framework
    \item Платформа разработки общего назначения
    \item Модульность
    \item С открытым исходным кодом
    \item Активно развивается
    \item Кроссплатформенность
    \begin{itemize}
        \item Windows, Linux, macOS
        \item x64, x86, ARM32, ARM64
    \end{itemize}
\end{itemize}
    
\end{frame}

\begin{frame}{Платформа .NET Core}
\begin{itemize}
    \item Производительность
    \item Библиотеки платформы .NET core
    \item C\#, Visual Basic, F\#
    \item CoreFX, CoreCLR
    \item Облочные и интернет технологии
\end{itemize}
    
\end{frame}

\begin{frame}{Немного про API}
\begin{itemize}
    \item Операция, которую можно выполнить
    \item Данные на вход
    \item Данные на выход
\end{itemize}
\end{frame}

\begin{frame}{Немного про API}
\begin{itemize}
    \item Примитивные типы
    \begin{itemize}
        \item System.Boolean, System.Int32
    \end{itemize}
    \item Коллекции
    \begin{itemize}
        \item System.Collections.Generic.List<T>, 
        \item System.Collections.Generic.Dictionary<TKey, TValue>
    \end{itemize}
    \item Служебные типы
    \begin{itemize}
        \item System.IO.FileStream
    \end{itemize}
    \item Высокопроизводительные типы
    \begin{itemize}
        \item System.Numerics.Vector, System.Span<T>
    \end{itemize}
\end{itemize}
\end{frame}

\begin{frame}{Преимущества .NET Core}
\begin{itemize}
    \item Мультиплатформенное ПО
    \item Для систем с высокоми требованиями производительности
    \item Для консольных приложений 
    \item При необходимости использования нескольких версий .NET
\end{itemize}
    
\end{frame}

\begin{frame}
\frametitle{Платформа Mono}
\begin{itemize}
    \item Изначатьно как реализация .NET на Linux
    \item Для разработки кроссплатформенных приложений
    \item Поддерживает стандарты C\# и стандарт CLI
\end{itemize}
\end{frame}

\begin{frame}
\frametitle{Компоненты Mono}
\begin{itemize}{}
    \item Компилятор языка C\#
    \item Среда исполнения Mono
    \begin{itemize}{}
        \item Common Language Runtime
        \item Компилятор CLI (Just-In-Time)
        \item Генератор машинного кода AOT (Ahead-Of-Time)
        \item Загрузчик сборок
        \item Сборщик мусора
        \item Подсистемы управления многопоточностью
        \item Компоненты поддержки взаимодействия между сборками и COM
    \end{itemize}
\end{itemize}
\end{frame}

\begin{frame}
\frametitle{Компоненты Mono}
\begin{itemize}
    \item Базовая библиотека классов
    \item Библиотека классов Mono
\end{itemize}
\end{frame}

\begin{frame}{Преимущества Mono}
    \begin{itemize}
        \item Мультиплатформенность
        \item Удобстово в разработке
        \item Некоторые библиотеки лучше работают под Mono
    \end{itemize}
\end{frame}
\end{document}


